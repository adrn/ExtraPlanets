\documentclass[letterpaper,12pt,preprint]{hack_aastex}

% Colors.
\usepackage{amsmath}
\usepackage{color}
\usepackage[pagebackref=false]{hyperref}
\definecolor{linkcolor}{rgb}{0,0,0.25}
\hypersetup{colorlinks=true,linkcolor=linkcolor,citecolor=linkcolor,
            filecolor=linkcolor,urlcolor=linkcolor}

\usepackage[para]{footmisc}
\urlstyle{sm}

\usepackage{epsfig}
\usepackage{graphicx}
\setlength{\headheight}{2ex}
\setlength{\headsep}{2ex}

\usepackage{multicol}

\newcommand{\dd}{\,\mathrm{d}}
\newcommand{\bvec}[1]{{\ensuremath{{\boldsymbol{#1}}}}}
\newcommand{\transpose}[1]{{#1}^{\!\mathsf{T}}}
\newcommand{\inverse}[1]{{#1}^{-1}}


%----- exact 1-in margins
% NB: headheight and headsep MUST exist and be set
\setlength{\textwidth}{6.5in}
\setlength{\textheight}{9in}
\addtolength{\textheight}{-1.0\headheight}
\addtolength{\textheight}{-1.0\headsep}
\setlength{\topmargin}{-2em}
\addtolength{\textheight}{-1.0\topmargin}
\setlength{\oddsidemargin}{0.0in}
\setlength{\evensidemargin}{0.0in}

%----- typeset certain kinds of words
\newcommand{\observatory}[1]{\textsl{#1}}
\newcommand{\package}[1]{\textsf{#1}}
\newcommand{\project}[1]{\package{#1}}
\newcommand{\an}{\package{Astrometry.net}}
\newcommand{\NASA}{\observatory{NASA}}
\newcommand{\JWST}{\observatory{JWST}}
\newcommand{\Kepler}{\observatory{Kepler}}
\newcommand{\kepler}{\Kepler}
\newcommand{\Ktwo}{\observatory{K2}}
\newcommand{\ktwo}{\Ktwo}
\newcommand{\KT}{\Ktwo}
\newcommand{\PLATO}{\observatory{PLATO}}
\newcommand{\MAST}{\observatory{MAST}}
\newcommand{\EA}{\observatory{Exoplanet Archive}}
\newcommand{\TESS}{\observatory{TESS}}
\newcommand{\galex}{\observatory{GALEX}}
\newcommand{\Spitzer}{\observatory{Spitzer}}
\newcommand{\gaia}{\observatory{Gaia}}
\newcommand{\Gaia}{\gaia}
\newcommand{\lsst}{\observatory{LSST}}
\newcommand{\sdss}{\observatory{SDSS}}
\newcommand{\latin}[1]{\textit{#1}}
\newcommand{\eg}{\latin{e.g.}}
\newcommand{\etal}{\latin{et~al.}}
\newcommand{\etc}{\latin{etc.}}
\newcommand{\ie}{\latin{i.e.}}
\newcommand{\vs}{\latin{vs.}}

\newcommand{\doi}[2]{\href{http://dx.doi.org/#1}{\emph{#2}}}
\newcommand{\ads}[2]{\href{http://adsabs.harvard.edu/abs/#1}{\emph{#2}}}
\newcommand{\arxiv}[1]{\href{http://arxiv.org/abs/#1}{arXiv:#1}}

%----- stuff for this proposal in particular.
\newcommand{\kplr}{\package{kplr}}
\newcommand{\PLM}{\package{PLM}}
\newcommand{\OWL}{\package{OWL}}
\newcommand{\George}{\package{George}}
\newcommand{\kpsf}{\package{kpsf}}
\newcommand{\emcee}{\package{emcee}}
\newcommand{\GitHub}{\package{GitHub}}
\newcommand{\KIC}{\textsl{KIC}}
\newcommand{\KOI}{\textsl{KOI}}

%----- typeset journals
% \newcommand{\aj}{Astron.\,J.}
% \newcommand{\apj}{Astrophys.\,J.}
% \newcommand{\apjl}{Astrophys.\,J.\,Lett.}
% \newcommand{\apjs}{Astrophys.\,J.\,Supp.\,Ser.}
% \newcommand{\mnras}{Mon.\,Not.\,Roy.\,Ast.\,Soc.}
% \newcommand{\aap}{Astron.\,\&~Astrophys.}

%----- Tighten up paragraphs and lists
\setlength{\parskip}{0.0ex}
\setlength{\parindent}{0.2in}
\renewenvironment{itemize}{\begin{list}{$\bullet$}{%
  \setlength{\topsep}{0.0ex}%
  \setlength{\parsep}{0.0ex}%
  \setlength{\partopsep}{0.0ex}%
  \setlength{\itemsep}{0.0ex}%
  \setlength{\leftmargin}{1.5\parindent}}}{\end{list}}
\newcounter{actr}
\renewenvironment{enumerate}{\begin{list}{\arabic{actr}.}{%
  \usecounter{actr}%
  \setlength{\topsep}{0.0ex}%
  \setlength{\parsep}{0.0ex}%
  \setlength{\partopsep}{0.0ex}%
  \setlength{\itemsep}{0.0ex}%
  \setlength{\leftmargin}{1.5\parindent}}}{\end{list}}

%----- mess with paragraph spacing!
\makeatletter
\renewcommand\paragraph{\@startsection{paragraph}{4}{\z@}%
                                    {1ex}%
                                    {-1em}%
                                    {\normalfont\normalsize\bfseries}}
\makeatother

%----- Special Hogg list for references
  \newcommand{\hogglist}{%
    \rightmargin=0in
    \leftmargin=0.25in
    \topsep=0ex
    \partopsep=0pt
    \itemsep=0ex
    \parsep=0pt
    \itemindent=-1.0\leftmargin
    \listparindent=\leftmargin
    \settowidth{\labelsep}{~}
    \usecounter{enumi}
  }

%----- side-to-side figure macro
%------- make numbers add up to 94%
 \newlength{\figurewidth}
 \newlength{\captionwidth}
 \newcommand{\ssfigure}[3]{%
   \setlength{\figurewidth}{#2\textwidth}
   \setlength{\captionwidth}{\textwidth}
   \addtolength{\captionwidth}{-\figurewidth}
   \addtolength{\captionwidth}{-0.02\figurewidth}
   \begin{figure}[htb]%
   \begin{tabular}{cc}%
     \begin{minipage}[c]{\figurewidth}%
       \resizebox{\figurewidth}{!}{\includegraphics{#1}}%
     \end{minipage} &%
     \begin{minipage}[c]{\captionwidth}%
       \textsf{\caption[]{\footnotesize {#3}}}%
     \end{minipage}%
   \end{tabular}%
   \end{figure}}

 \newcommand{\ssfiguretwo}[4]{%
   \setlength{\figurewidth}{#3\textwidth}
   \setlength{\captionwidth}{\textwidth}
   \addtolength{\captionwidth}{-\figurewidth}
   \addtolength{\captionwidth}{-0.02\figurewidth}
   \begin{figure}[htb]%
   \begin{tabular}{cc}%
     \begin{minipage}[c]{\figurewidth}%
       \resizebox{\figurewidth}{!}{\includegraphics{#1}}\\
       \resizebox{\figurewidth}{!}{\includegraphics{#2}}%
     \end{minipage} &%
     \begin{minipage}[c]{\captionwidth}%
       \textsf{\caption[]{\footnotesize {#4}}}%
     \end{minipage}%
   \end{tabular}%
   \end{figure}}

%----- top-bottom figure macro
 \newlength{\figureheight}
 \setlength{\figureheight}{0.75\textheight}
 \newcommand{\tbfigure}[2]{%
   \begin{figure}[htp]%
   \resizebox{\textwidth}{!}{\includegraphics{#1}}%
   \textsf{\caption[]{\footnotesize {#2}}}%
   \end{figure}}

%----- deal with pdf page-size stupidity
\special{papersize=8.5in,11in}
\setlength{\pdfpageheight}{\paperheight}
\setlength{\pdfpagewidth}{\paperwidth}

% no more bad lines!
\sloppy\sloppypar

% A better underline!
% tex.stackexchange.com/questions/36894/underline-omitting-the-descenders
\usepackage[T1]{fontenc}
\usepackage[latin1]{inputenc}
\usepackage{soul}
% \usepackage{xcolor}
\usepackage{xparse}
\makeatletter
\ExplSyntaxOn
\cs_new:Npn \white_text:n #1
  {
    \fp_set:Nn \l_tmpa_fp {.01}
    \fp_mul:Nn \l_tmpa_fp {#1}
    \llap{\textcolor{white}{\the\SOUL@syllable}\hspace{\fp_to_decimal:N \l_tmpa_fp em}}
    \llap{\textcolor{white}{\the\SOUL@syllable}\hspace{-\fp_to_decimal:N \l_tmpa_fp em}}
  }
\NewDocumentCommand{\whiten}{ m }
    {
      \int_step_function:nnnN {1}{1}{#1} \white_text:n
    }
\ExplSyntaxOff

\NewDocumentCommand{ \varul }{ D<>{5} O{0.2ex} O{0.1ex} +m } {%
\begingroup
\setul{#2}{#3}%
\def\SOUL@uleverysyllable{%
   \setbox0=\hbox{\the\SOUL@syllable}%
   \ifdim\dp0>\z@
      \SOUL@ulunderline{\phantom{\the\SOUL@syllable}}%
      \whiten{#1}%
      \llap{%
        \the\SOUL@syllable
        \SOUL@setkern\SOUL@charkern
      }%
   \else
       \SOUL@ulunderline{%
         \the\SOUL@syllable
         \SOUL@setkern\SOUL@charkern
       }%
   \fi}%
    \ul{#4}%
\endgroup
}
\makeatother


% Underline paragraph titles
\usepackage[explicit]{titlesec}
\titleformat{\paragraph}[runin]
    {\normalfont\normalsize\bfseries}{\theparagraph}{1em}
    {\varul{#1}}

\pagestyle{myheadings}
\markright{\textsf{\footnotesize %
                   The search for extragalactic exoplanets / %
                   Hogg, Price-Whelan \& Foreman-Mackey}}

% Single-spacing.
\def\baselinestretch{1.0}

\begin{document}

The Sagittarius (Sgr) stream is the most massive and prominent stellar tidal
stream observed in the halo of the Milky Way. Stars associated with the stream
(via kinematics and chemical abundances) fully encircle the Galaxy and span
heliocentric distances of $\approx$15 kpc to $\approx$100 kpc. The footprint
of \KT\ fields 8 and 10 overlap with regions of the stream previously detected
in K/M giant and F turnoff stars \citep{Majewski:2003, Yanny:2009} at
distances of $\approx$20--30 kpc. Though this is an order of magnitude farther
than any of the known transiting exoplanets, the stream is one of the closest
hosts of extragalactic stars --- thus, the Sgr stream represents a unique
opportunity to detect the first exoplanetary system formed outside of our own
Galaxy.

To detect transits at these large distances, bright (giant) targets are
required and only large planets on short orbital periods will be accessible.
Since few planetary systems orbiting giant stars have been discovered, little
is known about the occurrence rate and distribution of these planets (CITE).
Therefore, it is hard to make a robust prediction of the yield for this
proposal but we have demonstrated that it is possible to use \KT\ to discover
transiting planets around even the faintest targets
\citep{Foreman-Mackey:2015}.
The methods developed in trying to detect transits in the light curves of
these noisy faint stars will improve the detection efficiency of transit
search algorithms even when applied to brighter and quieter targets.

{\bf We propose to target the brightest likely members of the Sagittarius
stellar stream that fall in the fields-of-view of Campaigns 8 and 10 (K2C8 and
K2C10 respectively) and develop novel methods to search for hot Jupiters
transiting these stars.}

It has been demonstrated that the long cadence \KT\ light curves can be used
to measure the asteroseismic oscillations of giant stars (CITE Angus).
Applying this methodology to these proposed targets in the Sgr stream, {\bf we
propose to estimate the ages of giant stars tracing the stream and.}


\paragraph{Target selection}

We select M giant candidates using the Two Micron All Sky Survey \citep[2MASS;
][]{Skrutskie:2006} photometry provided in the \KT\ field
catalogs\footnote{\url{https://archive.stsci.edu/k2/catalogs.html}} following
the selection criteria used in \cite[][hereafter M03]{Majewski:2003}, with
slight modifications. We first use the dust reddening measurements of
\cite{Schlafly:2011} to de-redden the observed magnitudes (following the
conversion of E(B-V) to the 2MASS bandpasses used in M03); for this procedure,
we assume all stars are at sufficiently large distances and are therefore
behind the dust (an invalid assumption for dwarf stars). We select all point
source targets from this catalog by requiring that \texttt{Objtype == `STAR'}:
Figures~\ref{fig:field8} and \ref{fig:field10} show all point sources (black
markers) in color-magnitude (left) and color-color (right) space for field 8
and 10, respectively. To select the M giant candidates, we first require that
either the proper motion in any component is small ($|\mu_\alpha |,|\mu_\delta
| < 10~{\rm mas}~{\rm yr}^{-1}$) or that the proper motion is unmeasured. We
then use the following (unrestrictive) selection criteria to isolate M giant
candidates:
\begin{align} % get outta here with yr eqn arrays
	1.1 &> (J-K_s)_0 > 0.85 \\
	13.1 &> K_s > 10.5\\
	(J-H)_0 &< 0.561\,(J-K_s)_0 + 0.34\\
	(J-H)_0 &> 0.5\,(J-K_s)_0 + 0.2\\
	(J-H)_0 &> -0.4\,(J-K_s)_0 + 1.1
\end{align}
where the first color cut is designed to select M type stars, the $K_s$
magnitude limits are designed to only select would-be M giants in the distance
range $\approx$15--40 kpc, and the other color cuts are to limit contamination
from dwarf stars \citep[e.g.,][]{Majewski:2003}. For selection in field 10, we
instead use the magnitude range $14 > K_s > 12$ because the stream is more
distant ($d_\odot \approx 60$ kpc). Over-plotted on
Figures~\ref{fig:field8}--\ref{fig:field10} in larger (blue) markers are the
targets that pass these criteria, along with the selection window (in
color-color space). There are 46 targets in field 8 and 202 targets in field
10. Because the targets in field 10 are fainter, there is more contamination
with dwarfs.

\tbfigure{fig1.pdf}{%
Color-magnitude (left) and color-color (right) plot for all 2MASS
stars in Field 8. M giant candidates must match color selection region (shown
between thick lines) and must either have low proper motion or no measured
proper motions --- candidates are shown as larger (blue) markers.
\label{fig:field8}}

\tbfigure{fig2.pdf}{%
Same as Figure~\ref{fig:field8} but for Field 10.
\label{fig:field10}}

We have received time on the 2.4 meter Hiltner telescope at MDM Observatory to
obtain spectra and therefore radial velocities for a sample of these stars.
With radial velocities we will be able to (a) remove M dwarf contaminants and
(2) further purify the sample to select M giants that are kinematically
associated with the Sgr stream.

\paragraph{Transit search methodology}

We have developed a method for searching for transit signals using the \KT\
data products and this method resulted in the first systematic catalog of
transiting exoplanets from \KT\ \citep{Foreman-Mackey:2015}.
This method is robust to the systematic variability of the light curves
introduced by instrumental effects but it is susceptible to overfitting when
applied to the light curves of intrinsically variable stars
\citep{Montet:2015}.
Since giant stars have higher amplitude variability, it is necessary to
augment the light curve noise model with a term accounting for the
contribution from the intrinsic variability of the star.
It has been demonstrated that a Gaussian Process \citep[GP;][]{Rasmussen:2006,
Ambikasaran:2014} is a good effective model for this stochastic variability in
the light curves of giant stars \citep{Barclay:2015}.
In our original transit search procedure, we phrased the problem as linear
regression \citep{Foreman-Mackey:2015} where the light curve $\bvec{f}$ is
modeled as
\begin{eqnarray}
\bvec{f} &=& \bvec{B}\,\bvec{w} + \bvec{m} + \mathrm{noise}
\end{eqnarray}
where $\bvec{B}$ is a set of eigen light curves (ELCs), $\bvec{m}$ is a
transit model, and the linear weights $\bvec{w}$ are marginalized out in the
computation.
This (marginalized) model can be written equivalently as the GP
\citep{Rasmussen:2006}
\begin{eqnarray}
\bvec{f} &\sim& \mathcal{N} (\bvec{m},\,
\bvec{\Sigma} + \bvec{B}\bvec{\Lambda}\bvec{B}^\mathrm{T})
\end{eqnarray}
where $\bvec{\Sigma}$ is the diagonal matrix of measurement variances and
$\bvec{\Lambda}$ is the diagonal matrix of the prior variances on the ELC
weights.
This means that a temporal model for the stochastic stellar variability can
also be included
\begin{eqnarray}
\bvec{f} &\sim& \mathcal{N} (\bvec{m},\,
\bvec{\Sigma} + \bvec{B}\bvec{\Lambda}\bvec{B}^\mathrm{T} + \bvec{K})
\end{eqnarray}
where the elements $K_{ij}$ of $\bvec{K}$ are given by the covariance function
$k(t_i,\,t_j)$ following \citep{Barclay:2015}.

{\bf We propose to generalize our previously published transit search method
\citep{Foreman-Mackey:2015} to account for the variability of giant stars and
use this procedure to search for exoplanets transiting giant stellar members
of the Sgr stellar stream.} The proposed methodological developments have
implications beyond this proposed goal and could be used for robustly
detecting transiting planets orbiting variable stars in existing time series
datasets (\KT\ and \kepler, for example) and upcoming missions like \TESS\ and
\PLATO.

\paragraph{Estimated planet yield} blah.

\paragraph{The age distribution of the Sagittarius stream}

Out of ideas here! Maybe drop it?



\begin{multicols}{2}
{\centering\bf REFERENCES\par}
\vspace{0.2em}
\begin{thebibliography}{}%
\raggedright\raggedbottom\scriptsize\setlength{\parskip}{-0.5em}%

\bibitem[Ambikasaran \etal(2014)]{Ambikasaran:2014}
Ambikasaran, S., Foreman-Mackey, D., Greengard, L., Hogg, D.~W.,
\& O'Neil, M.\ 2014, \arxiv{1403.6015}

\bibitem[Barclay \etal(2015)]{Barclay:2015}
Barclay, T., Endl, M., Huber, D., Foreman-Mackey, D., \etal\ 2015, \apj, 800,
46

\bibitem[Foreman-Mackey \etal(2014)]{Foreman-Mackey:2014}
Foreman-Mackey, D., Hogg, D.~W., \& Morton, T.~D.\ 2014, \apj, 795, 64

\bibitem[Foreman-Mackey \etal(2015)]{Foreman-Mackey:2015}
Foreman-Mackey, D., Montet, B.~T., Hogg, D.~W., \etal\ 2015, \arxiv{1502.04715}

\bibitem[Majewski \etal(2003)]{Majewski:2003}
Majewski, S.~R., Skrutskie, M.~F., Weinberg, M.~D., \& Ostheimer, J.~C.\ 2003,
\apj, 599, 1082

\bibitem[Montet \etal(2015)]{Montet:2015}
Montet, B.~T., Morton, T.~D., Foreman-Mackey, D., \etal\ 2015,
\arxiv{1503.07866}

\bibitem[Rasmussen \& Williams(2006)]{Rasmussen:2006}
Rasmussen, C.~E., \& Williams, C.~K.~I., \emph{Gaussian Processes for Machine
Learning}, The MIT Press, 2006

\bibitem[Schlafly \& Finkbeiner(2011)]{Schlafly:2011}
Schlafly, E.~F., \& Finkbeiner, D.~P.\ 2011, \apj, 737, 103

\bibitem[Skrutskie \etal(2006)]{Skrutskie:2006}
Skrutskie, M.~F., Cutri, R.~M., Stiening, R., \etal\ 2006, \aj, 131, 1163

\bibitem[Yanny \etal(2009)]{Yanny:2009}
Yanny, B., Newberg, H.~J., Johnson, J.~A., \etal\ 2009, \apj, 700, 1282

\end{thebibliography}
\end{multicols}

\end{document}
