\documentclass[letterpaper,12pt,preprint]{hack_aastex}

\input{hogg_nasa}
\pagestyle{myheadings}
\markright{\textsf{\footnotesize %
                   The search for extragalactic exoplanets / %
                   Hogg, Price-Whelan \& Foreman-Mackey}}

% Single-spacing.
\def\baselinestretch{1.0}

\begin{document}

The current.

{\bf We propose to target the brightest likely members of the Sagittarius
stellar stream that fall in the field-of-view of Campaign 8 and search for hot
Jupiters transiting these stars.}

\paragraph{Target selection}

\paragraph{Transit search methodology}

We have developed a method for searching for transit signals using the \KT\
data products and this method resulted in the first systematic catalog of
transiting exoplanets from \KT\ \citep{Foreman-Mackey:2015}.
This method is robust to the systematic variability of the light curves
introduced by instrumental effects but it is susceptible to overfitting when
applied to the light curves of intrinsically variable stars
\citep{Montet:2015}.
Since giant stars have higher amplitude variability, it is necessary to
augment the light curve noise model with a term accounting for the
contribution from the intrinsic variability of the star.
It has been demonstrated that a Gaussian Process
\citep[GP;][]{Rasmussen:2006, Ambikasaran:2014} is a good effective
model for this stochastic variability \citep{Barclay:2015}.
In our original transit search procedure, we phrased the problem as linear
regression \citep{Foreman-Mackey:2015} where the light curve $\bvec{f}$ is
modeled as
\begin{eqnarray}
\bvec{f} &=& \bvec{B}\,\bvec{w} + \bvec{m} + \mathrm{noise}
\end{eqnarray}
where $\bvec{B}$ is a set of eigen light curves (ELCs), $\bvec{m}$ is a
transit model, and the linear weights $\bvec{w}$ are marginalized out in the
computation.
This model can equivalently be written as a GP \citep{Rasmussen:2006}
\begin{eqnarray}
\bvec{f} &\sim& \mathcal{N} (\bvec{m},\,
\bvec{\Sigma} + \bvec{B}\bvec{\Lambda}\bvec{B}^\mathrm{T})
\end{eqnarray}
where $\bvec{\Sigma}$ is the diagonal matrix of measurement variances and
$\bvec{\Lambda}$ is the diagonal matrix of the prior variances on the ELC
weights.
This means that a temporal model for the stochastic stellar variability can
also be included
\begin{eqnarray}
\bvec{f} &\sim& \mathcal{N} (\bvec{m},\,
\bvec{\Sigma} + \bvec{B}\bvec{\Lambda}\bvec{B}^\mathrm{T} + \bvec{K})
\end{eqnarray}
where the elements $K_{ij}$ of $\bvec{K}$ are given by the covariance function
$k(t_i,\,t_j)$.

\paragraph{Estimated planet yield} blah.

\paragraph{The age distribution of the Sagittarius stream}

blah


% \ssfigure{../figures/period.pdf}{0.4}{%
% The posterior probability density for the period
% of a Neptune-sized transiting planet with a $300\,\mathrm{day}$ orbital period
% injected into the \Kepler\ light curve for a relatively bright (\Kepler\
% magnitude $13.5$) G-dwarf.
% \emph{Top:} The light curve---de-trended using a median filter after the
% injection---centered on the transit.
% The data are shown as black points and the lines are $50$ posterior
% predictions for light curve.
% \emph{Bottom:} The Markov Chain Monte Carlo estimate of the posterior
% constraint on the period of the orbit given this single transit.
% The vertical gray line indicates the true injected period.
% To produce this measurement, I assumed that the stellar mass and radius were
% known to $10\%$ and chose a beta function prior on the eccentricity
% \citep{ecc-prior}.
% This distribution is covariant with the stellar parameters, eccentricity, and
% impact parameter and in this figure, I marginalized over these effects.
% \label{fig:period}}

\begin{multicols}{2}
{\centering\bf REFERENCES\par}
\vspace{0.2em}
\begin{thebibliography}{}%
\raggedright\raggedbottom\scriptsize\setlength{\parskip}{-0.5em}%

\bibitem[Ambikasaran \etal(2014)]{Ambikasaran:2014}
Ambikasaran, S., Foreman-Mackey, D., Greengard, L., Hogg, D.~W.,
\& O'Neil, M.\ 2014, \arxiv{1403.6015}

\bibitem[Barclay \etal(2015)]{Barclay:2015}
Barclay, T., Endl, M., Huber, D., Foreman-Mackey, D., \etal\ 2015, \apj, 800,
46

\bibitem[Foreman-Mackey \etal(2014)]{Foreman-Mackey:2014}
Foreman-Mackey, D., Hogg, D.~W., \& Morton, T.~D.\ 2014, \apj, 795, 64

\bibitem[Foreman-Mackey \etal(2015)]{Foreman-Mackey:2015}
Foreman-Mackey, D., Montet, B.~T., Hogg, D.~W., \etal\ 2015, \arxiv{1502.04715}

\bibitem[Montet \etal(2015)]{Montet:2015}
Montet, B.~T., Morton, T.~D., Foreman-Mackey, D., \etal\ 2015,
\arxiv{1503.07866}

\bibitem[Rasmussen \& Williams(2006)]{Rasmussen:2006}
Rasmussen, C.~E., \& Williams, C.~K.~I., \emph{Gaussian Processes for Machine
Learning}, The MIT Press, 2006

\end{thebibliography}
\end{multicols}

\end{document}
